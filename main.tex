\documentclass[12pt,a4paper,twoside]{article}

% Core packages
\usepackage[utf8]{inputenc}
\usepackage[T1]{fontenc}
\usepackage[english,arabic,chinese]{babel}
\usepackage{geometry}
\usepackage{amsmath,amsfonts,amssymb,amsthm}
\usepackage{mathtools}
\usepackage{graphicx}
\usepackage{booktabs}
\usepackage{array}
\usepackage{longtable}
\usepackage{multirow}
\usepackage{multicol}
\usepackage{enumerate}
\usepackage{fancyhdr}
\usepackage{hyperref}
\usepackage{xcolor}
\usepackage{listings}
\usepackage{algorithm}
\usepackage{algpseudocode}
\usepackage{subcaption}
\usepackage{float}
\usepackage{wrapfig}
\usepackage{lipsum}

% Advanced graphics and plotting
\usepackage{tikz}
\usepackage{pgfplots}
\usepackage{tikz-3dplot}

% TikZ libraries
\usetikzlibrary{
    3d,
    arrows,
    arrows.meta,
    calc,
    decorations.markings,
    decorations.pathreplacing,
    matrix,
    patterns,
    positioning,
    shapes,
    shapes.geometric,
    shapes.misc,
    datavisualization,
    datavisualization.formats.functions
}

% PGFPlots configuration
\pgfplotsset{compat=1.18}
\usepgfplotslibrary{
    colormaps,
    fillbetween,
    statistics,
    polar,
    ternary,
    dateplot,
    patchplots
}

% Multi-language support setup
\usepackage{CJKutf8}
\usepackage{arabtex}
\usepackage{utf8}

% Page geometry
\geometry{
    left=2.5cm,
    right=2.5cm,
    top=3cm,
    bottom=3cm,
    headheight=15pt
}

% Header and footer
\pagestyle{fancy}
\fancyhf{}
\fancyhead[LE,RO]{\thepage}
\fancyhead[LO]{\rightmark}
\fancyhead[RE]{\leftmark}
\fancyfoot[C]{Ultra-Advanced LaTeX Test}

% Theorem environments
\newtheorem{theorem}{Theorem}[section]
\newtheorem{lemma}[theorem]{Lemma}
\newtheorem{proposition}[theorem]{Proposition}
\newtheorem{corollary}[theorem]{Corollary}
\theoremstyle{definition}
\newtheorem{definition}[theorem]{Definition}
\newtheorem{example}[theorem]{Example}

% Code listings setup
\lstset{
    backgroundcolor=\color{gray!10},
    basicstyle=\ttfamily\footnotesize,
    breakatwhitespace=false,
    breaklines=true,
    captionpos=b,
    commentstyle=\color{green!60!black},
    keywordstyle=\color{blue},
    stringstyle=\color{red!80!black},
    numbers=left,
    numbersep=5pt,
    numberstyle=\tiny\color{gray},
    frame=single,
    tabsize=2,
    showspaces=false,
    showstringspaces=false
}

% Custom commands
\newcommand{\R}{\mathbb{R}}
\newcommand{\N}{\mathbb{N}}
\newcommand{\Z}{\mathbb{Z}}
\newcommand{\Q}{\mathbb{Q}}
\newcommand{\C}{\mathbb{C}}
\newcommand{\abs}[1]{\left|#1\right|}
\newcommand{\norm}[1]{\left\|#1\right\|}

% Hyperref setup
\hypersetup{
    colorlinks=true,
    linkcolor=blue,
    urlcolor=blue,
    citecolor=red,
    pdftitle={Ultra-Advanced LaTeX Test},
    pdfauthor={GitHub Actions Compiler}
}

\title{\textbf{Ultra-Advanced LaTeX Feature Test}\\
       \large 3D Graphics, Multi-Language \& Complex Visualization}
\author{GitHub Actions Automated Compiler}
\date{\today}

\begin{document}

\maketitle

\begin{abstract}
This document pushes the boundaries of LaTeX compilation by incorporating advanced TikZ 3D graphics, complex mathematical plots, multi-language text rendering, and sophisticated data visualization. It serves as the ultimate stress test for our GitHub Actions workflow with cached TeX Live installation.
\end{abstract}

\tableofcontents
\newpage

\section{Advanced 3D Graphics with TikZ}

\subsection{3D Coordinate Systems}

\begin{figure}[H]
\centering
\tdplotsetmaincoords{70}{110}
\begin{tikzpicture}[scale=3,tdplot_main_coords]
    % Draw the main coordinate system
    \draw[thick,->] (0,0,0) -- (1.5,0,0) node[anchor=north east]{$x$};
    \draw[thick,->] (0,0,0) -- (0,1.5,0) node[anchor=north west]{$y$};
    \draw[thick,->] (0,0,0) -- (0,0,1.5) node[anchor=south]{$z$};
    
    % Draw a 3D cube
    \draw[blue,thick] (0,0,0) -- (1,0,0) -- (1,1,0) -- (0,1,0) -- cycle;
    \draw[blue,thick] (0,0,1) -- (1,0,1) -- (1,1,1) -- (0,1,1) -- cycle;
    \draw[blue,thick] (0,0,0) -- (0,0,1);
    \draw[blue,thick] (1,0,0) -- (1,0,1);
    \draw[blue,thick] (1,1,0) -- (1,1,1);
    \draw[blue,thick] (0,1,0) -- (0,1,1);
    
    % Draw a 3D vector
    \draw[red,very thick,->] (0,0,0) -- (0.8,0.6,0.9) node[above]{$\vec{v}$};
    
    % Draw some points
    \fill[red] (0.8,0.6,0.9) circle (1pt);
    \fill[green] (0.5,0.5,0.5) circle (1pt);
    
    % Add grid on xy-plane
    \foreach \x in {0,0.25,0.5,0.75,1}
        \foreach \y in {0,0.25,0.5,0.75,1}
            \fill[gray,opacity=0.3] (\x,\y,0) circle (0.5pt);
\end{tikzpicture}
\caption{3D Coordinate System with Cube and Vector}
\label{fig:3d-coords}
\end{figure}

\subsection{Complex 3D Surface}

\begin{figure}[H]
\centering
\begin{tikzpicture}
\begin{axis}[
    view={60}{30},
    width=12cm,
    height=8cm,
    xlabel=$x$,
    ylabel=$y$,
    zlabel=$z$,
    title={3D Surface: $z = \sin(x) \cos(y) e^{-(x^2+y^2)/4}$},
    colormap/cool,
    shader=interp
]
\addplot3[
    surf,
    domain=-2:2,
    domain y=-2:2,
    samples=25,
    samples y=25
] {sin(deg(x)) * cos(deg(y)) * exp(-(x^2 + y^2)/4)};
\end{axis}
\end{tikzpicture}
\caption{Complex 3D Mathematical Surface}
\label{fig:3d-surface}
\end{figure}

\section{Advanced Data Visualization}

\subsection{Statistical Plots}

\begin{figure}[H]
\centering
\begin{subfigure}{0.48\textwidth}
\centering
\begin{tikzpicture}
\begin{axis}[
    width=\textwidth,
    height=6cm,
    ybar,
    xlabel={Categories},
    ylabel={Frequency},
    title={Histogram with Error Bars},
    symbolic x coords={A,B,C,D,E},
    xtick=data,
    error bars/.cd,
        y dir=both,
        y explicit
]
\addplot+[error bars/.cd, y dir=both, y explicit] 
coordinates {
    (A,20) +- (0,2)
    (B,35) +- (0,3)
    (C,45) +- (0,4)
    (D,28) +- (0,2)
    (E,38) +- (0,3)
};
\end{axis}
\end{tikzpicture}
\caption{Bar Chart with Error Bars}
\end{subfigure}
\hfill
\begin{subfigure}{0.48\textwidth}
\centering
\begin{tikzpicture}
\begin{axis}[
    width=\textwidth,
    height=6cm,
    xlabel={Time (s)},
    ylabel={Signal Amplitude},
    title={Multi-Signal Plot},
    legend pos=north east,
    grid=major
]
\addplot[blue, thick, smooth] table {
    x y
    0 0
    1 0.841
    2 0.909
    3 0.141
    4 -0.757
    5 -0.959
    6 -0.279
    7 0.657
    8 0.989
    9 0.412
    10 -0.544
};
\addlegendentry{$\sin(x)$}

\addplot[red, thick, smooth] table {
    x y
